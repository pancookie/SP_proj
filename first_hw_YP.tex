% First thing first, you can put your comments after the percentage symbol 
% multiple comment lines by starting with %
% how about comment line section?

% so all the comment stuff would at the line before the commands

% to create an article, class options in squres
% intialized an article doc, enlarged fontsize for readability for the elder (older than me)
% papersize is old-school a4 sheet, print-friendly
\documentclass[14pt, letterpaper]{article} 

% reservation for preambles, I compare it as 'import' in python, to use additional modules
% allow the use of memes
\usepackage{graphicx}
% of coursely I love to mark the line no.
\usepackage{lineno}
% package for multi-line comment
\usepackage{comment}
% package for strike-through effect. DO NOT MY MIND!
\usepackage[normalem]{ulem}
% extra command for citation
\usepackage[authoryear]{natbib}
% to use python within latex
\usepackage{listings}
% to use color effect on the text background; enable highlight color
\usepackage{xcolor, soul}

% starting the document
\begin{document}
% line spacing
\linespread{1.66667}

\begin{comment}
Nonsense words came in here and nobody should see it 
Gold AD slot for rent
contact: (817)-291-8139

For each begin, there is always an end. (best to my knowledge I am unsure if there is some fancy commands different)
\end{comment}

% I need a title page
\title{HOMEWORK - 1: \\for now I still prefer MS words more}
% culprit information I am glad my name is very neat and simple
\author{Yang Pan}
% include the current time
\date{\today}

% confirm the title page
\maketitle
% start a new page
\newpage


% here comes the table of contents (ToC) section
% it would handle everything automatically 
% [faith in Latex restored]
\tableofcontents
% of course we leave the ToC in individual pages
\newpage

% turn on line numbering
\linenumbers

% start the section 
\section{Programming experiences}
I was overwhelmed with programming classes in my Mater degree in computer applications.
% \textbf{} to bold the text within the curly braces
And I have studied various languages such as \textbf{C, C++, C\# (the C family), Java}, etc. \textbf{Python} was never put 
into the syllabus, but I studied it on my own since them. Meanwhile, you have to study many other languages 
such as \textbf{HTML, CSS, PHP, JavaScript} those language in web technologies. It was scary to use out-dated WAMP 
server since the programming exams were all performed on the system of the department. Also, database management 
system is fun to play with.
% put new line as double backslashes for new line
\\ \\
Currently, I am typing the script on the \textbf{MacOS} system in my lab, with Mojave installed (luckily the system prevented 
me from upgrading the OS months ago). My personal Windows laptop Dell G7 is used for \sout{gaming} parallel working experience 
when I am \sout{listening to music} doing researches in the lab. 
\\
I prefer \textbf{terminal} on Mac since \sout{Mac is much more expensive than Linux laptop} MacOS is smooth to use. 
When talking about editor, I prefer \textbf{Sublime} for the code viewing. If it is the file at remote server, \textbf{vim} is the 
tool I use instead of emacs. Without no offense to emacs at all. 
\textbf{Anaconda} is the favorite IDE I use for data handling (graph drawing, stats calculation) with python script. I also install 
anaconda to remote server to install packages in really good handling.

\newpage


\section{Physics equation}
The Gauss's Law in differential form:

% start the module of equation
\begin{equation}
%add label for ref purpose
	\label{eqn:gauss_law}
	\vec{\nabla} \cdot \vec{E}  =  \frac{\rho}{\varepsilon_0}
\end{equation}
% use \ref for cross reference
And for the details of the parameters in the Equation \ref{eqn:gauss_law} \\
	
% write a table to explain the equation, and forcefully (!) to place the table "here"
\begin{table}[!h]
	% centralize the table
	\begin{center}
	% create tabular module, | for the frames
	\begin{tabular}{|l|c|}
		% the content of first row
		% horizontal line
		\hline
		Variable & Description \\ 
		\hline
		% specific contents of the table
		$\vec{E}$ & Electric field \\ 
		\hline
		$\rho$ & Density of charges enclosed \\ 
		\hline
		$\varepsilon_0$ & Electric constant \\
		\hline
	\end{tabular}
	% don't forget the caption
	\caption{Gauss's Law differential form}
	\end{center}
\end{table}
The Equation \ref{eqn:gauss_law} shows the divergence of electric field equals to the ratio between density of 
charge enclosed and the electric constant. 
This equation is very neat in providing the way to calculate electric field by 
focusing on the charge density enclosed! I have met it a lot when doing the homeworks and especially during 
the exams. Kind of \sout{PTSD} awesome. 

\newpage


\section{Comment}
It is a very detailed tutorial including some many useful demonstrations in using the modules. \sout{Thus, I have
not finished reading all the parts.} \\
Some suggestions might go there: \\
\noindent 1. It looks better if we centralize the front page vertically; \\
\noindent 2. Can we be taught how to include GIFs in \LaTeX?

\newpage


\section{Related works}
% \cite{} to cite the reference in .bib files
% \citet for citation author-year without braces
I am going to \sout{copy-and-paste} refer from the paper I have written. \\
Total Electron Content (TEC) is an important parameter characterizing the ionospheric
plasma number density \citep{mannucci1998global}. Especially, the dynamic TEC value can be used
to identify travelling ionospheric disturbance, which indicates magnetic storm events. TEC also
influences the communication between satellites and the ground stations and has been included
as a parameter in the space weather forecasting \citep{jakowski2002gps,afraimovich2008tec}. 
And we have used the deep convolutional generative adversarial network (DCGAN) to train the International 
Global Navigation Satellite System (GNSS) Service (IGS)-TEC data with the post processing Poisson blending, and it outperforms conventional 
image inpainting methods \citet{pan2020tec}. Sadly, it will not increase my citation.

\newpage

\section{More Memes}
Frankly speaking, I really \sout{learned a lot} collected a lot of memes from this class. And here is an 
absolutely good chance for me to put more. \\
So this is my favorite cartoon character shown in Figure \ref{fig:pooh}, really adorable, isn't it? \\

% to add the image, must be put at bottom of the page
\begin{figure}[!b]
	% similar to what we have done for table, centralize it.
	\begin{center}
	%specify the width and name of image
	\includegraphics[width=12cm]{pooh_pants.png}
	% put the caption
	\caption{That is a bad idea.}
	% label it for ref purpose
	\label{fig:pooh}
	\end{center}
\end{figure}
\newpage

\section{Show me something new}
Unlikely, I was trying to include a GIF into this pdf but both movie15 and media9 have failed. (Faith in \LaTeX 
dropped $\Downarrow$). % double down arrow
% create a subsection
\subsection{This is a Python class}
As you can see below, I am adding my python script file with a better visualization effect \ref{lst:python_1}.\\
To add the coding block, we are going to use a lot of \begin{verbatim} \lstset{} \end{verbatim} commands. 
So type the following in your .tex file: \\

\nolinenumbers
\begin{verbatim}
\lstset{language=Python} % set the programming language
\lstset{frame=lines} % you are going to put multiple lines
\lstset{caption={A simple python example}} % caption is important
\lstset{label={lst:python_1}} % you don't want to refer to it?
\lstset{basicstyle=\small} % fontsize

\begin{lstlisting}
	[your code section]
\end{lstlisting}
\end{verbatim}
\linenumbers
And it goes like below: 
% specify the language
\lstset{language=Python}
% it will include multiple lines
\lstset{frame=lines}
% caption the block
\lstset{caption={A simple python example}}
% label it for reference
\lstset{label={lst:python_1}}
% define the font size of the block
\lstset{basicstyle=\small}
% start the coding block
\begin{lstlisting}
import numpy as np

yang = np.arange(-180, 180.1, 5)

for item in yang:
	print(item)
\end{lstlisting}

\subsection{Coloured text}
To enable this module, import the package \texttt{xcolor} as 
\verb+usepackage{xcolor}+. % \verb for inline verbatim environment
Then \verb+\textcolor{color}{Text to be colored}+ to \textcolor{red}{color} \textcolor{yellow}{the} 
\textcolor{blue}{text} \textcolor{orange}{you} \textcolor{violet}{want}. \\

\subsection{Highlightened}
You could import \texttt{xcolor} module along with \texttt{soul} module as \verb+\usepackage{xcolor, soul}+. 
Use \verb+\hl{the highlighted part}+ for the command. If we do not specify the color we want, by default \hl{yellow would be applied}. Name the color before use \verb+\hl{}+, with \verb+\sethlcolor{unicorn_color}+. 
\sethlcolor{green}
\hl{How are doing Ms.~Brown?}



\newpage





% let it automatically restrict the length of one row in para? like ctrl+J in MS word

% new stuff to include
% display name in diff format
% examine typo, I sensed there is even no typo detector? Faith in \latex down to 0

% mandatory for a new page
\clearpage
% to show bibliography in table of content
\addcontentsline{toc}{section}{Bibliography}
% citation style file, the built-in style
\bibliographystyle{plainnat}
% citation .bib file. It tells the filename of bib file.
\bibliography{hw1_YP}

% DO NOT forget me!
\end{document}